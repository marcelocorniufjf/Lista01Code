\documentclass[portuguese,12pt,a4paper]{article}
\usepackage[T1]{fontenc}
\usepackage{hyperref}
\usepackage{listings}
\usepackage{babel}
\title{Lista 01 de exercícios AED}
\author{Marcelo Corni Alves}
\begin{document}
	\maketitle

\begin{center}
	Códigos das implementações da lista de exercícios disponíveis em
	 \textbf{\url{https://github.com/marcelocorniufjf/Lista01Code}}
\end{center}


\section*{Listas, Pilhas e Filas}

\paragraph{1-)}
Para cada estrutura abaixo, implemente os métodos contido(K,L), inserir(K,L) e remover(K,L) e verifique as complexidades de cada método.

No caso de filas e pilhas, o remover(K,L) não terá o argumento K, visto que filas e pilhas removem sempre quem está na extremidade.

\paragraph{a-)}
Lista duplamente encadeada: considere a inserção sempre no final da lista

\lstset{language=C++,
	basicstyle=\footnotesize,
	numbers=left,
	numberstyle=\footnotesize,
	frame=shadowbox}
\begin{lstlisting} 
public class No
{
  public int chave;
  public No proximo;
  public No anterior;

  public No(int chave)
  {
    this.chave = chave;
    proximo = null;
    anterior = null;
  }
}

public class ListaDuplamenteEncadeada
{
  private No cabeca;
  private No cauda;

  public ListaDuplamenteEncadeada()
  {
    cabeca = null;
    cauda = null;
  }

  public bool Contido(int chave)
  {
    No atual = cabeca;
    while (atual != null)
    {
      if (atual.chave == chave)
      {
        return true;
      }
      atual = atual.proximo;
    }
    return false;
  }

  public void Inserir(int chave)
  {
    No novoNo = new No(chave);
    if (cabeca == null)
    {
      cabeca = novoNo;
      cauda = novoNo;
    }
    else
    {
      cauda.proximo = novoNo;
      novoNo.anterior = cauda;
      cauda = novoNo;
    }
  }

  public void Remover(int chave)
  {
    No atual = cabeca;
    while (atual != null)
    {
      if (atual.chave == chave)
      {
        if (atual.anterior != null)
        {
          atual.anterior.proximo = atual.proximo;
        }
        else
        {
          cabeca = atual.proximo;
        }
        if (atual.proximo != null)
        {
          atual.proximo.anterior = atual.anterior;
        }
        else
        {
          cauda = atual.anterior;
        }
        return;
      }
      atual = atual.proximo;
    }
  }
}

\end{lstlisting}

\paragraph{Complexidades:}
	A complexidade de tempo para Contido(K,L) é O(n), para Inserir(K,L) é O(1), e para Remover(L) é O(n), onde n é o número de elementos na lista.
	

\paragraph{b-)}
Fila com lista simplesmente encadeada: considere que vc tem uma variável head e uma tail, onde head marca a cabeça e tail o ultimo nó na cauda.


\lstset{language=C++,
	basicstyle=\footnotesize,
	numbers=left,
	numberstyle=\footnotesize,
	frame=shadowbox}
\begin{lstlisting} 
public class No
{
  public int chave;
  public No proximo;

  public No(int chave)
  {
    this.chave = chave;
    this.proximo = null;
  }
}

public class Fila
{
  private No cabeca;
  private No cauda;

  public Fila()
  {
    this.cabeca = null;
    this.cauda = null;
  }

  public bool Contido(int chave)
  {
    No atual = this.cabeca;
    while (atual != null)
    {
      if (atual.chave == chave)
      {
        return true;
      }
      atual = atual.proximo;
    }
    return false;
  }

  public void Inserir(int chave)
  {
    No novoNo = new No(chave);
    if (this.cabeca == null)
    {
      this.cabeca = novoNo;
      this.cauda = novoNo;
    }
    else
    {
      this.cauda.proximo = novoNo;
      this.cauda = novoNo;
    }
  }

  public void Remover()
  {
    if (this.cabeca != null)
    {
      this.cabeca = this.cabeca.proximo;
      if (this.cabeca == null)
      {
        this.cauda = null;
      }
    }
  }
}
\end{lstlisting}

\paragraph{Complexidades:}
A complexidade de tempo para Contido(K,L) é O(n), para Inserir(K,L) é O(1), e para Remover(L) é O(1), onde n é o número de elementos na fila.


\paragraph{c-)}
Pilha com lista simplesmente encadeada

\lstset{language=C++,
	basicstyle=\footnotesize,
	numbers=left,
	numberstyle=\footnotesize,
	frame=shadowbox}
\begin{lstlisting} 
public class No
{
  public int chave;
  public No proximo;

  public No(int chave)
  {
    this.chave = chave;
    this.proximo = null;
  }
}

public class Pilha
{
  private No topo;

  public Pilha()
  {
    this.topo = null;
  }

  public bool Contido(int chave)
  {
    No atual = this.topo;
    while (atual != null)
    {
      if (atual.chave == chave)
      {
        return true;
      }
      atual = atual.proximo;
    }
    return false;
  }

  public void Inserir(int chave)
  {
    No novoNo = new No(chave);
    if (this.topo == null)
    {
      this.topo = novoNo;
    }
    else
    {
      novoNo.proximo = this.topo;
      this.topo = novoNo;
    }
  }

  public void Remover()
  {
    if (this.topo != null)
    {
      this.topo = this.topo.proximo;
    }
  }
}
\end{lstlisting}	

\paragraph{Complexidades:}
A complexidade de tempo para Contido(K,L) é O(n), para Inserir(K,L) é O(1), e para Remover(L) é O(1), onde n é o número de elementos na pilha.
	
\paragraph{d-)}
Lista duplamente encadeada circular

\lstset{language=C++,
	basicstyle=\footnotesize,
	numbers=left,
	numberstyle=\footnotesize,
	frame=shadowbox}
\begin{lstlisting} 
public class No
{
  public int chave;
  public No proximo;
  public No anterior;

  public No(int chave)
  {
    this.chave = chave;
    this.proximo = null;
    this.anterior = null;
  }
}

public class ListaDuplamenteEncadeadaCircular
{
  private No cabeca;
  private No cauda;

  public ListaDuplamenteEncadeadaCircular()
  {
    this.cabeca = null;
    this.cauda = null;
  }

  public bool Contido(int chave)
  {
    No atual = this.cabeca;
    
    
    while (atual != null && atual != this.cauda)
    {
      if (atual.chave == chave)
      {
        return true;
      }
      atual = atual.proximo;
    }
    if (this.cauda != null && this.cauda.chave == chave)
    {
      return true;
    }
    return false;
  }

  public void Inserir(int chave)
  {
    No novoNo = new No(chave);
    if (this.cabeca == null)
    {
      this.cabeca = novoNo;
      this.cauda = novoNo;
      novoNo.proximo = novoNo;
      novoNo.anterior = novoNo;
    }
    else
    {
      novoNo.proximo = this.cabeca;
      novoNo.anterior = this.cauda;
      this.cabeca.anterior = novoNo;
      this.cauda.proximo = novoNo;
      this.cauda = novoNo;
    }
  }

  public void Remover(int chave)
  {
    No atual = this.cabeca;
    while (atual != null && atual != this.cauda)
    {
      if (atual.chave == chave)
      {
        atual.anterior.proximo = atual.proximo;
        atual.proximo.anterior = atual.anterior;
        if (atual == this.cabeca)
        {
          this.cabeca = atual.proximo;
        }
        
        if (atual == this.cauda)
        {
          this.cauda = atual.anterior;
        }
       return;
      }
      atual = atual.proximo;
    }
    if (this.cauda != null && this.cauda.chave == chave)
    {
      this.cauda.anterior.proximo = this.cabeca;
      this.cabeca.anterior = this.cauda.anterior;
      this.cauda = this.cauda.anterior;
    }
  }
}
\end{lstlisting}		

\paragraph{Complexidades:}
A complexidade de tempo para Contido(K,L) é O(n), para Inserir(K,L) é O(1), e para Remover(L) é O(n) no pior caso (quando o elemento a ser removido está no final da lista).


\paragraph{2-)}
Listas são usadas para representar números muito grandes (p.ex, com 1000 dígitos), uma vez que seria impossível representá-lo em máquinas de 64bits. 

Para representar inteiros grandes com listas, é usada uma representação em que cada dígito do inteiro é armazenado em um nó da lista.\newline

Considere duas listas encadeadas L1 e L2 representando números grandes (cada 
digito por nó).

Faça um algoritmo que faça a soma de dois inteiros grandes e retorne a lista L3 = L1 + L2.

\lstset{language=C++,
	basicstyle=\footnotesize,
	numbers=left,
	numberstyle=\footnotesize,
	frame=shadowbox}
\begin{lstlisting} 
public class No
{
  public int digito;
  public No proximo;

  public No(int digito)
  {
    this.digito = digito;
    this.proximo = null;
  }
}

public class ListaInteiroGrande
{
  private No cabeca;

  public ListaInteiroGrande()
  {
    this.cabeca = null;
  }

  public void InserirDigito(int digito)
  {
    No novoNo = new No(digito);
    if (this.cabeca == null)
    {
      this.cabeca = novoNo;
    }
    else
    {
      novoNo.proximo = this.cabeca;
      this.cabeca = novoNo;
    }
  }

  public ListaInteiroGrande Soma(ListaInteiroGrande lista1,
                                 ListaInteiroGrande lista2)
  {
    ListaInteiroGrande resultado = new ListaInteiroGrande();
    No atual1 = lista1.cabeca;
    No atual2 = lista2.cabeca;
    int somatorio = 0;

    while (atual1 != null || atual2 != null || somatorio != 0)
    {
      int soma = somatorio;
      if (atual1 != null)
      {
        soma += atual1.digito;
        atual1 = atual1.proximo;
      }
      if (atual2 != null)
      {
        soma += atual2.digito;
        atual2 = atual2.proximo;
      }
      somatorio = soma / 10;
      int digitoSoma = soma % 10;
      resultado.InserirDigito(digitoSoma);
    }
    return resultado;
  }

  public void Imprimir()
  {
    No atual = this.cabeca;
    while (atual != null)
    {
      Console.Write(atual.digito);
      atual = atual.proximo;
    }
    Console.WriteLine();
  }
}
\end{lstlisting}	

\paragraph{3-)}
Seja L uma lista simplesmente encadeada. Escreva um algoritmo que, percorrendo a lista uma única vez, constrói:

A. Uma lista L’ que possui os valores de L em ordem inversa

B. Uma lista L’' que possui a metade dos nós da lista L, onde o primeiro nó de L’' contém a soma do primeiro nó de L com o último nó de L, o segundo nó de L’ contém a soma do segundo nó de L com o penúltimo nó de L e assim por diante: L''= < L1+Ln, L2+Ln-1, L3+Ln-2, ... , Ln/2 + Ln/2+1>, onde n é sempre par.\newline

\lstset{language=C++,
	basicstyle=\footnotesize,
	numbers=left,
	numberstyle=\footnotesize,
	frame=shadowbox}
\begin{lstlisting} 
public class No
{
  public int Valor { get; set; }
  public No Proximo { get; set; }

  public No(int valor)
  {
    Valor = valor;
    Proximo = null;
  }
}

public class ListaEncadeada
{
  private No cabeca;
  private int contador;

  public ListaEncadeada()
  {
    cabeca = null;
    contador = 0;
  }

  public void Inserir(int valor)
  {
    No novoNo = new No(valor);
    if (cabeca == null)
    {
      cabeca = novoNo;
    }
    else
    {
      No atual = cabeca;
      while (atual.Proximo != null)
      {
        atual = atual.Proximo;
      }
      atual.Proximo = novoNo;
    }
    contador++;
  }

  public void Imprimir()
  {
    No atual = cabeca;
    while (atual != null)
    {
      Console.Write(atual.Valor + " ");
      atual = atual.Proximo;
    }
    Console.WriteLine();
  }

  public ListaEncadeada InverterLista()
  {
    ListaEncadeada listaReversa = new ListaEncadeada();
    No atual = cabeca;
    while (atual != null)
    {
      listaReversa.InserirNaFrente(atual.Valor);
      atual = atual.Proximo;
    }
    return listaReversa;
  }

  public ListaEncadeada SomarMeiaLista()
  {
    ListaEncadeada somarMeiaLista = new ListaEncadeada();
    No atual = cabeca;
    for (int i = 0; i < contador / 2; i++)
    {
    	
      int sum = atual.Valor 
                + PegarNoNaPosicao(contador - i - 1).Valor;
      somarMeiaLista.Inserir(sum);
      atual = atual.Proximo;
    }
    return somarMeiaLista;
  }

  private void InserirNaFrente(int valor)
  {
    No novoNo = new No(valor);
    novoNo.Proximo = cabeca;
    cabeca = novoNo;
    contador++;
  }

  private No PegarNoNaPosicao(int posicao)
  {
    if (posicao < 0 || posicao >= contador)
    {
      throw new ArgumentOutOfRangeException(nameof(posicao));
    }

    No atual = cabeca;
    for (int i = 0; i < posicao; i++)
    {
      atual = atual.Proximo;
    }
    return atual;
  }
}
\end{lstlisting}	

\paragraph{4-)}
Escreva um algoritmo para reconhecer se uma dada palavra é um palíndromo. 

Considere que a palavra está contida em uma lista simplesmente encadeada, onde cada caractere está em um nó da lista.


\lstset{language=C++,
	basicstyle=\footnotesize,
	numbers=left,
	numberstyle=\footnotesize,
	frame=shadowbox}
\begin{lstlisting} 
public class No
{
  public char Valor { get; set; }
  public No Proximo { get; set; }

  public No(char value)
  {
    Valor = value;
    Proximo = null;
  }
}

public class ListaEncadeada
{
  private No cabeca;
  private int contador;

  public ListaEncadeada()
  {
    cabeca = null;
    contador = 0;
  }

  public void Inserir(char value)
  {
    No novoNo = new No(value);
    if (cabeca == null)
    {
      cabeca = novoNo;
    }
    else
    {
      No atual = cabeca;
      while (atual.Proximo != null)
      {
        atual = atual.Proximo;
      }
      atual.Proximo = novoNo;
    }
    contador++;
  }

  public bool EPalindromo()
  {
    No lenta = cabeca;
    No rapida = cabeca;
    Stack<char> pilha = new Stack<char>();

    while (rapida != null && rapida.Proximo != null)
    {
      pilha.Push(lenta.Valor);
      lenta = lenta.Proximo;
      rapida = rapida.Proximo.Proximo;
    }

    if (rapida != null)
    {
      lenta = lenta.Proximo;
    }

    while (lenta != null)
    {
      if (pilha.Pop() != lenta.Valor)
      {
        return false;
      }
      lenta = lenta.Proximo;
    }

    return true;
  }
}
\end{lstlisting}


\paragraph{5-)}
Seja A uma matriz esparsa n x m.
\paragraph{a-)}
Crie uma estrutura de dados que represente A e cujo espaço total seja O(k) em vez de O(mn), onde k é o número total de elementos não irrelevantes de A.

\paragraph{b-)}
Faça um algoritmo para localizar um valor $a_{ij}$ na estrutura acima.
\paragraph{c-)}
Faça um algoritmo para computar $A^2$ utilizando a estrutura acima. Para tal, crie os algoritmos de inserção e busca na sua estrutura.
\paragraph{d-)}
Qual a complexidade da sua solução para a letra (c)?
\newline\newline
\textbf{R:} O($k^2$), onde k é o número total de elementos não irrelevantes da matriz esparsa.

\paragraph{5.5-)}
Caso não tenha pensado numa estrutura com duas listas (uma para colunas e uma para
linhas), recomenda-se refazer o exercício acima com essa forma de estruturação.\newline\newline\newline\newline

\textbf{Solução para letras a-), b-) e c-):}

\lstset{language=C++,
	basicstyle=\footnotesize,	
	numbers=left,
	numberstyle=\footnotesize,
	frame=shadowbox}
\begin{lstlisting} 
public class MatrizEsparsa
{
  // Estrutura de dados para representar a matriz esparsa
  private List<int> linhas;
  private List<int> colunas;
  private List<int> valores;

  // Construtor para inicializar a matriz esparsa
  public MatrizEsparsa()
  {
    linhas = new List<int>();
    colunas = new List<int>();
    valores = new List<int>();
  }

  // a) Metodo para inserir um valor na matriz
  public void Inserir(int linha, int coluna, int valor)
  {
    linhas.Add(linha);
    colunas.Add(coluna);
    valores.Add(valor);
  }

  // b) Metodo para localizar um valor na matriz
  public int Procurar(int linha, int coluna)
  {
    for (int i = 0; i < linhas.Count; i++)
    {
      if (linhas[i] == linha && colunas[i] == coluna)
      {
        return valores[i];
      }
    }
    return 0; // Valor irrelevante se nao encontrado
  }

  // c) Metodo para calcular A^2 utilizando a estrutura
  public MatrizEsparsa Quadrado()
  {
    MatrizEsparsa result = new MatrizEsparsa();
    for (int i = 0; i < linhas.Count; i++)
    {
      int linhaAtual = linhas[i];
      int colunaAtual = colunas[i];
      int valorAtual = valores[i];
      for (int j = 0; j < linhas.Count; j++)
      {
        if (colunaAtual == linhas[j])
        {
          int proximaColuna = colunas[j];
          int proximoValor = valores[j];
          int produto = valorAtual * proximoValor;
          result.Inserir(linhaAtual, proximaColuna, produto);
        }
      }
    }
    return result;
  }

  // Metodo para imprimir a matriz esparsa
  public void Imprimir()
  {
    for (int i = 0; i < linhas.Count; i++)
    {
      Console.WriteLine("({0}, {1}): {2}", 
                                linhas[i], 
                               colunas[i], 
                              valores[i]);
    }
  }
}
\end{lstlisting}


\paragraph{}
\textbf{Código para execução de testes dos códigos da lista de exercícios}\newline\newline

\lstset{language=C++,
	basicstyle=\footnotesize,	
	numbers=left,
	numberstyle=\footnotesize,
	frame=shadowbox}
\begin{lstlisting} 
Console.WriteLine("Codigos da Lista 01 de exercicios de AED");

//Questao 01 - Letra A - Exemplos de uso

var q01A = new Lista01Code.Questao01.LetraA
               .ListaDuplamenteEncadeada();

q01A.Inserir(10);
q01A.Inserir(5);
q01A.Inserir(6);
q01A.Inserir(55);

var procurar01AChave6 = q01A.Contido(6);

Console.WriteLine(procurar01AChave6);

//Remove a chave 6
q01A.Remover(6);

procurar01AChave6 = q01A.Contido(6);

Console.WriteLine(procurar01AChave6);


//Questao 01 - Letra B - Exemplos de uso

var q01B = new Lista01Code.Questao01.LetraB.Fila(); //FIFO

q01B.Inserir(10);
q01B.Inserir(5);
q01B.Inserir(6);
q01B.Inserir(55);

var procurar01BChave10 = q01B.Contido(10);

Console.WriteLine(procurar01BChave10);


//Remove a chave 10
q01B.Remover();

procurar01BChave10 = q01B.Contido(10);
Console.WriteLine(procurar01BChave10);


//Questao 01 - Letra C - Exemplo de uso

var q01C = new Lista01Code.Questao01.LetraC.Pilha(); //LIFO

q01C.Inserir(10);
q01C.Inserir(5);
q01C.Inserir(6);
q01C.Inserir(55);

var procurar01CChave55 = q01C.Contido(55);

Console.WriteLine(procurar01CChave55);


//Remove a chave 55
q01C.Remover();

procurar01CChave55 = q01C.Contido(55);
Console.WriteLine(procurar01CChave55);

//Questao 01 - Letra D - Exemplo de uso

var q01D = new Lista01Code.Questao01.LetraD
               .ListaDuplamenteEncadeadaCircular();

q01D.Inserir(10);
q01D.Inserir(5);
q01D.Inserir(6);
q01D.Inserir(55);

var procurar01DChave55 = q01D.Contido(55);

Console.WriteLine(procurar01DChave55);

//Remove a chave 55
q01D.Remover(55);

procurar01DChave55 = q01D.Contido(55);
Console.WriteLine(procurar01DChave55);


//Questao 02 - Exemplo de uso

var q02Lista1 = new Lista01Code.Questao02.ListaInteiroGrande();

var q02Lista2 = new Lista01Code.Questao02.ListaInteiroGrande();

var q02Lista3 = new Lista01Code.Questao02.ListaInteiroGrande();

q02Lista1.InserirDigito(9);
q02Lista1.InserirDigito(9);
q02Lista1.InserirDigito(9);

q02Lista2.InserirDigito(5);
q02Lista2.InserirDigito(1);
q02Lista2.InserirDigito(2);

//Somar as listas - Resultado esperado 1511 = (999 + 512)
q02Lista3 = q02Lista3.Soma(q02Lista1, q02Lista2);
q02Lista3.Imprimir();


//Questao 03 - Exemplo de uso
var q03 = new Lista01Code.Questao03.ListaEncadeada();

q03.Inserir(10);
q03.Inserir(5);
q03.Inserir(6);
q03.Inserir(55);

q03.Imprimir();

q03 = q03.InverterLista();

q03.Imprimir();

var q03Soma = new Lista01Code.Questao03.ListaEncadeada();

q03Soma = q03.SomarMeiaLista();

q03Soma.Imprimir();

//Questao 04 - Exemplo de uso

var q0401 = new Lista01Code.Questao04.ListaEncadeada();

q0401.Inserir('R');
q0401.Inserir('A');
q0401.Inserir('D');
q0401.Inserir('A');
q0401.Inserir('R');

Console.WriteLine(q0401.EPalindromo() ? 
                       "E palindromo" : 
                   "Nao e Palindromo");

var q0402 = new Lista01Code.Questao04.ListaEncadeada();

q0402.Inserir('B');
q0402.Inserir('A');
q0402.Inserir('T');
q0402.Inserir('A');
q0402.Inserir('T');
q0402.Inserir('A');

Console.WriteLine(q0402.EPalindromo() ? 
                       "E palindromo" : 
                   "Nao e Palindromo");


//Questao 05 - Exemplo de uso

var q05 = new Lista01Code.Questao05.MatrizEsparsa();

// a) Inserindo valores na matriz
q05.Inserir(0, 0, 2);
q05.Inserir(0, 1, 3);
q05.Inserir(0, 2, 4);
q05.Inserir(1, 0, 5);
q05.Inserir(1, 1, 6);
q05.Inserir(1, 2, 7);

Console.WriteLine("Matriz Esparsa:");
q05.Imprimir();


// b) Localizando um valor na matriz
int valor = q05.Procurar(1, 0);
Console.WriteLine("Valor encontrado: " + valor);

// c) Calculando A^2
var q05Quadrado = q05.Quadrado();
Console.WriteLine("Matriz A^2:");
q05Quadrado.Imprimir();

// d) complexidade O(k^2) do medodo Quadrado(), sendo K 
o numero de elementos nao irrelevantes.

Console.ReadKey();

\end{lstlisting}

\end{document}